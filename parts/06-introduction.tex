\section*{\centering ВВЕДЕНИЕ}
\addcontentsline{toc}{section}{ВВЕДЕНИЕ}

В современном мире компьютерная графика окружает человека практически везде.
Прежде всего стал популярен синтез изображений, так как с помощью него разработчики могут создавать компьютерные игры, спецэффекты в кино, виртуальную реальность. Компьютерная графика используется в науке и промышленности для визуализации и
моделирования различных процессов.

В мире компьютерной графики, одним из ключевых аспектов, влияющих на визуальное восприятие и качество изображений, является процесс наложения текстур.
Эта техника позволяет придавать поверхностям и объектам на экране уникальный внешний вид, подчеркивать их характер и структуру, а также сделать изображения более реалистичными и привлекательными для наблюдателя.

Целью данной работы является разработка программного обеспечения для наложения текстур на трёхмерные объекты.

Для достижения поставленной цели необходимо решить следующие задачи:
\begin{itemize}
	\item провести анализ существующих алгоритмов компьютерной графики, используемых для создания трехмерных сцен;
	\item выбрать алгоритмы для решения поставленной задачи;
	\item выбрать язык программирования и среду разработки для реализации поставленной задачи;
\end{itemize}